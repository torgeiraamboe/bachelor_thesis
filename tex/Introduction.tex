\section*{Introduction}
%
In topology, the notion of \emph{homotopy} characterizes the similarity of functions on topological spaces. A topological homotopy uses a topological space to say when two functions between topological spaces can be deformed into each other. 
It also characterizes when two spaces are similar enough to be continuously deformed into each other, called homotopy equivalence. A lot of the topological invariants often studied in topology are unchanged by homotopies, i.e. invariant under homotopy, making it a very strong and useful tool. 

Furthermore, using the equivalence classes generated by homotopy, this creates the notion of homotopy categories. The archetypal example is topological spaces together with homotopy classes of continuous maps. These are in fact such strong tools that mathematicians want to make an equivalent theory in other areas of mathematics. 

The classical method for doing so is to generalize the theory. This usually involves making the theory categorical and then applying it to other categories than the standard one we started with. This has been done with homotopy theory and the most used and recognized generalizations are called \emph{model categories} and \emph{\((\infty, 1)\)-categories}. The former is a purely categorical theory, while in the latter one needs higher category theory. 
We are not going to go deeper into these in this thesis, but we will be studying a special or simple case of the notion of homotopy in model categories. \\


We are going to explore a first approach to create a homotopy theory for schemes, called \emph{naive algebraic homotopy theory}. The general theory of the homotopy theory discussed in this thesis is called \emph{motivic homotopy theory} or \(\mathbb{A}^1\)-homotopy theory. 
The general theory is part of the generalization of homotopy theory to model categories, and one constructs the \(\mathbb{A}^1\)-homotopy category as the homotopy category of a model category for this theory. This construction is outside the scope of this thesis, but it can be useful to make a connection to this thesis if one later in life studies motivic homotopy theory and want intuition of something a little closer to home (which is of course normal topological homotopy theory). \\


The construction of our naive homotopy theory will be based heavily on the usual construction in topology, in the way that we change out the unit interval with something we can actually use in algebraic geometry. 
The unit interval is not an algebraic variety, so in the context of algebraic geometry we don't know how to use it. We are going to switch it out with the algebro-geometric version of the real line, or in general the simplest 1-dimensional scheme over a field, namely the \emph{affine line} \(\mathbb{A}^1\). This yields an interesting equivalence relation, however, it doesn't lead to the correct homotopy category for schemes. Nevertheless, Cazanave proves in \cite{Cazanave} that this naive version has some interesting properties when looking at the naive homotopy classes of endomorphisms of \(\mathbb{P}^1\). Cazanave proves that these classes admit a commutative monoid structure, and that the canonical morphism \([\mathbb{P}^1, \mathbb{P}^1]^N \to [\mathbb{P}^1, \mathbb{P}^1]^{\mathbb{A}^1} \) is a group completion. In this thesis we study the first result, that the homotopy classes form a commutative monoid, and we present a more detailed and slightly modified proof. 
\bigskip \\

The thesis is split into two chapters, each of them having their own sections. We give a short overview on the different parts of the thesis.

\textbf{Chapter 1} is titled naive homotopy and the scheme of rational functions, and this chapter consists mostly of definitions and comparisons. In this chapter we are defining and creating the building blocks for the rest of the thesis. We define what a naive homotopy is, what the scheme of rational functions is, what we mean by k-scheme morphisms, the projective line and more. 

\textbf{Chapter 2} is titled: Naive homotopy classes as a commutative monoid, and this chapter is where we prove the main theorem of the thesis. We start by constructing a monoid operation, and then show that the naive homotopy classes defined in chapter 1, together with this operation form a commutative monoid. 
\bigskip \\

For this thesis \(k\) is always a field with characteristic not equal to \(2\). Any ring mentioned is always a commutative unital ring. 
 



